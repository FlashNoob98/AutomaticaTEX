\section{Progettazione di un controllore}

Sia dato un sistema come quello in figura \ref{fig:sistema_generico}, la
cui funzione di trasferimento è:
$$
    G(s) = \frac{30}{(s+10)(s+3)} = \frac{1}{\left(\frac{s}{10}+1\right)\left(\frac{s}{3}+1\right)}
$$
si vuole
progettare un controllore $K(s)$ in modo tale che il sistema a ciclo chiuso
rispetti le seguenti specifiche:
\begin{itemize}
    \item $e(t)|_{\infty} = 0 \ |\ r(t) = 1(t) $ errore a regime nullo con ingresso a gradino $\rightarrow K_1(s) = \frac{1}{s}$
    \item Attenuazione $ATT \geq 95\%\ \omega \in [100\ +\infty[\ \rightarrow |F(j\omega_d)| \geq \SI{-26}{\deci\bel}$
    \item Sovraelongazione $s\leq 20\%$
    \item Tempo di assestamento $t_{s,1\%} \leq 0.5 s$
\end{itemize}

Il primo procedimento è quello di tradurre le specifiche nei diagrammi di Bode,
innanzitutto per annullare l'errore con ingresso a graino è necessario avere
un polo nell'origine, dopo essersi assicurati che questo non sia già presente nel sistema è possibile aggiungere il primo elemento del controllore parziale $K'(s) = \frac{1}{s}$.

Garantire un'attenuazione del 95\% significa avere un sistema che abbia modulo pari a $1-0.95$ con ingresso unitario, superata la pulsazione critica si assume che la funzione $F(s)$ e la funzione $T(s)$ abbiano lo stesso andamento, dunque
è sufficiente inserire nel piano di Bode dei moduli un'area di non attraversammento della $F(s)$ per pulsazioni maggiori di \SI{100}{\radian/\second} e moduli maggiori di $20\log_{10}(1-0.95) = \SI{-26}{\deci{\bel}}$.

La sovraelongazione $s$ ci dà informazioni riguardo la tipologia di funzione approssimante se del primo o secondo ordine e il margine di fase da rispettare,
dalla teoria si sa che dato un valore massimo di sovraelongazione (ad esempio 20\%)
$$
    s = e^{\frac{-\pi \zeta}{\sqrt{1-\zeta^2}}} \leq 20\% \Rightarrow \zeta \geq \sqrt{\frac{\ln s^2}{\pi^2 + \ln s^2}}
$$
in questo caso $\zeta \geq 0.456$, con buona approssimazione si stima il margine
di fase desiderato pari a $100/\zeta = 45.6^\circ$, di conseguenza essendo tale margine minore di \SI{60}{\degree} si approssima il sistema a ciclo chiuso ad uno del secondo ordine del tipo
$$
    T_a(s) =  \frac{1}{1+\frac{2\zeta}{\omega_n}s + \frac{s^2}{\omega_n^2}}
$$
la curva inviluppo della risposta ha un andamento del tipo
$$
    e^{-\zeta \omega_n t_s}
$$
con $\omega_n$ la pulsazione naturale e $t_s$ il tempo di assestamento, se si considera ad esempio un assestamento all'1\% in un tempo di \SI{0.5}{\second} come in questo caso si ricava la pulsazione naturale
$$
    e^{-\zeta \omega_n t_s} \leq 1\% \Rightarrow -\zeta\omega_n t_s \leq \ln{1\%} = -4.6 \Rightarrow \omega_n \geq \frac{4.6}{\zeta t_s} \geq \frac{4.6}{0.456\cdot 0.5} \geq \SI{20}{\radian/\second}
$$

Ricavate tutte le specifiche è possibile iniziare la progettazione del controllore.

Si riporta sui diagrammi di Bode la funzione di trasferimento del sistema con
il polo aggiunto nell'origine, rispettando la prima specifica.
$$
    F'(s) = \frac{1}{s}\frac{1}{\left(\frac{s}{10}+1\right)\left(\frac{s}{3}+1\right)}
$$

\begin{figure}[h]
    \centering
    \begin{tikzpicture}[]
        \begin{scope}[xscale=7/2,yscale=3/110]
            \tikzset{
                semilog lines/.style={black},
            }
            \UnitedB
            \OrdBode{30}
            \semilog{-1}{2}{-90}{30}
            \BodeGraph[asymp lines,samples=400]
            {-1:2}{\IntAmp{1}+\POAmpAsymp{1}{0.1}+\POAmpAsymp{1}{0.333}
            }
            \BodeGraph[samples=400]{-1:2}{\IntAmp{1}+\POAmp{1}{0.1}+\POAmp{1}{0.333}}
        \end{scope}
        \begin{scope}[xscale=7/2,yscale=3/90]
            \tikzset{
                semilog lines/.style={black},
            }
            \UniteDegre
            \OrdBode{15}
            \semilog{-1}{2}{-270}{-90}
            \BodeGraph[asymp lines,samples=400]
            {-1:2}{\IntArg{1}+\POArg{1}{0.1}+\POArg{1}{0.333}
            }
            \BodeGraph[samples=400]{-1:2}{\IntArg{5}+\POArgAsymp{1}{0.1}+\POArgAsymp{1}{0.333}}
        \end{scope}
    \end{tikzpicture}
    \caption{Diagrammi di Bode della funzione di trasferimento con polo}
    \label{fig:bode_es_2}
\end{figure}

Analizzando i diagrammi di Bode si nota come il sistema non rispetti il punto
di attraversamento a \SI{20}{\radian\per\second} nè abbia un margine di fase
di $45.6^\circ$ a tale pulsazione, si vede come in questo caso la specifica sulla fase (ovvero sulla sovraelongazione) sia molto più critica rispetto a quella sulla pulsazione naturale (ovvero il tempo di assestamento) dunque si decide di procedere prima al recupero della fase.
Tracciando i diagrammi asintotici per le fasi si può osservare come questa sia pari a circa $-235^\circ$ in $\omega=\SI{20}{\radian/\second}$ mentre ne sono richiesti $45.6^\circ$ di margine dunque rispetto alla fase $-180^\circ$, il recupero necessario è facilmente ricavabile con la seguente:
$$
    \Delta\varphi(\omega_d) = 180 + \angle{\varphi'(\omega_d)} - \varphi_{md}
$$
in questo caso (approssimando 45.6 a 45) $\angle{\varphi'(\omega_d)} = -235$ e $\varphi_{md} = 45$ si ottiene $\Delta\varphi(\omega_d) = -100$, vanno recuperati 100 gradi.

Esistono diverse soluzioni al problema, bisogna capire inizialmente la struttura del controllore per poi determinare i punti in cui posizionare poli e zeri, si sa che ogni zero anticipa la fase di 90 gradi, dunque ne saranno necessari almeno due per un guadagno di 100, uno zero andrà sicuramente inserito prima della pulsazione critica, il secondo potrebbe essere inserito prima o dopo, prima di procedere al posizionamento degli zeri però si nota che così facendo si avrebbe una pendenza residua sul diagramma dei moduli pari al massimo a -1, ovvero se si impone il passaggio in zero a \SI{20}{\radian/\second} con una pendenza massima di -1 si arriverebbe alla pulsazione di \SI{100}{\radian/\per\second} con un modulo pari a \SI{-20}{\deci\bel}, dalle specifiche di attenuazione del rumore però sono richiesti almeno \SI{-26}{\deci\bel}, dunque la pendenza di attraversamento del punto critico deve essere maggiore, è allora necessario inserire almeno un polo nel controllore.

Si ipotizza di inserire uno zero centrato in 1, lontano almeno una decade dal punto critico che permetta di recuperare i primi 90 gradi, restano da recuperare
ulteriori 10 gradi con il rimanente zero e il polo, se si posizionano ad un terzo di decade di distanza questi daranno un incremento di circa 15 gradi in un intervallo di una decade e mezzo, si può facilmente vedere con i diagrammi asintotici delle fasi.
Si sceglie dunque di posizionare il secondo zero in $\omega=5$ e il polo in $\omega = 10$, la funzione di trasferimento della seconda parte di controllore sarà:
$$
    K''(s) = \frac{(s+1)(s/5+1)}{s/10+1}
$$
\newpage
\begin{figure}[h]
    \centering
    \begin{tikzpicture}[]
        \begin{scope}[xscale=7/2,yscale=3/60]
            \tikzset{
                semilog lines/.style={black},
            }
            \UnitedB
            \OrdBode{10}
            \semilog{-1}{2}{-10}{50}
            \BodeGraph[asymp lines,samples=400]
            {-1:2}{\POAmpAsymp{1}{0.1}-\POAmpAsymp{1}{1}-\POAmpAsymp{1}{0.2}
            }
            \BodeGraph[samples=400]{-1:2}{\POAmp{1}{0.1}-\POAmp{1}{1}-\POAmp{1}{0.2}
            }
        \end{scope}
        \begin{scope}[yshift=-4cm,xscale=7/2,yscale=3/120]
            \tikzset{
                semilog lines/.style={black},
            }
            \UniteDegre
            \OrdBode{20}
            \semilog{-1}{2}{-10}{110}
            \BodeGraph[samples=400]
            {-1:2}{\POArg{1}{0.1}-\POArg{1}{1}-\POArg{1}{0.2}
            }
        \end{scope}
    \end{tikzpicture}
    \caption{Diagrammi del controllore $K''(s)$}
    \label{fig:bode_K2_es_2}
\end{figure}

Anche senza rappresentare i diagrammi di Bode si nota che l'aggiunta di poli e
zeri implica anche una variazione del modulo, si hanno \SI{+26}{\deci\bel}
dovuti dallo zero in 1 che si trova dunque ad una decade più un terzo di decade
prima di 20, \SI{+13.3}{\deci\bel} dovuti allo zero in 5, a due terzi di decade
di distanza e \SI{-6.66}{\deci\bel} dovuti al polo in 10, un terzo di decade
prima, sommando tutti i contributi si ottiene un guadagno in $\omega=20$ pari a
circa \SI{32}{\deci\bel}, andavano recuperati circa \SI{50}{\deci\bel} vuol
dire che c'è una differenza di ancora \SI{18}{\deci\bel}, possono essere
recuperati con il parametro $K$, eseguendo la formula inversa del logaritmo
$$
    K = 10^{\frac{18}{20}} \simeq 8.16
$$