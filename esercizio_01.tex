
\section{Specifiche a regime}


\begin{tikzpicture}[auto, node distance=2cm,>=latex']
    % We start by placing the blocks
    \node [input, name=input] {};
    \node [sum, right of=input] (sum) {};
    \node [block, right of=sum] (controller) {$K(s)$};
    \node [block, right of=controller,
            node distance=3cm] (system) {$G(s)$};
    % We draw an edge between the controller and system block to
    % calculate the coordinate u. We need it to place the measurement block.
    \draw [->] (controller) -- node[name=u] {$U$} (system);
    \node [sum, right of=system, node distance=2cm] (disturbance) {};
    %\node [block, above of=disturbance] (g_d) {$G_d(s)$};
    \node [input, above of=disturbance] (dist_input){$D$};
    \node [output, right of=disturbance] (output) {};
    \node [block, below of=u] (measurements) {$G_m(s)$};

    % Once the nodes are placed, connecting them is easy.
    \draw [draw,->] (input) -- node {$R(s)$} (sum);
    \draw [->] (sum) -- node {$E(s)$} (controller);
    \draw [->] (system) -- node {$Y_u$} (disturbance);
    \draw [->] (disturbance) -- node [name=y] {$Y$}(output);
    %\draw [->] (g_d) -- node {$Y_d$} (disturbance);
%    \draw [draw, ->] (dist_input) -- node {$D$} (g_d);
    %\draw [->] (y) |- (measurements);
    \draw [->] (measurements) -| node[pos=0.99] {$-$}
        node [near end] {$Y_m$} (sum);
\end{tikzpicture}
