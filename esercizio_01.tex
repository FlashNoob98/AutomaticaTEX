
\section{Specifiche a regime}

\begin{figure}[h]\centering
    \begin{tikzpicture}[auto, node distance=2cm,>=latex']
        % We start by placing the blocks
        \node [input, name=input] at (0,0) {};
        \node [sum] (sum) at (1,0) {};
        \node [block] (controller) at (3.5,0) {$K(s)$};
        \node [block] (system) at (6.5,0) {$G(s)$};
        % We draw an edge between the controller and system block to
        % calculate the coordinate u. We need it to place the measurement block.
        \draw [->] (controller) -- node[name=u] {$U(s)$} (system);
        \node [sum, ] (disturbance) at (8.5,0) {};
        %\node [input, above of=disturbance] (g_d) {$G_d(s)$};
        \node [input] (dist_input) at (8.5,1) {$D(s)$};
        \node [output] (out2) at (9,0) {};
        \node [sum, ] (noise) at (9,-1.5) {};
        \node [output, right of=disturbance] (output) {};
        \node [input] (noise_in) at (10,-1.5) {};
        %\node [block, below of=u] (measurements) {$GAAA(s)$};


        % Once the nodes are placed, connecting them is easy.
        \draw [->] (input) -- node {$R(s)$} (sum);
        \draw (input) -- node[pos=0.9,yshift=-13] {$+$}(sum);
        \draw [->] (sum) -- node {$E(s)$} (controller);
        \draw [->] (system) -- node[pos=0.9,yshift=-13] {$+$} (disturbance);
        \draw [->] (disturbance) -- node [name=y,xshift=14] {$Y(s)$}(output);

        %\draw [->] (g_d) -- node {$Y_d$} (disturbance);
        \draw [draw, ->] (dist_input) node[xshift=-13,yshift=-5]{$D(s)$} -- node[pos=0.8,xshift=-13] {$+$} (disturbance);
        %\draw [->] (y) |- (measurements);
        \draw [->] (out2) -- node[pos=0.9,xshift=-13] {$+$} node{}(noise);
        %\draw [->] (noise_in) -- node{$N(s)$}(noise);
        \draw [->] (noise) -| node[pos=0.95,xshift=13] {$-$}  node [near end] {} (sum);
        \draw [->] (noise_in) -- node[pos=1] {$+$}  node [label={$N(s)$}] {} (noise);
    \end{tikzpicture}
    \caption[]{Sistema generico}
    \label{fig:sistema_generico}
\end{figure}

In figura \ref{fig:sistema_generico} è possibile visualizzare lo schema elementare di un sistema generico con controllore in retroazione, si indicano in tabella \ref{tab:elementi_sistema} i riferimenti ai vari componenti del sistema, tutte le funzioni si assumono presentate nel dominio di Laplace.
\begin{table}[h]\centering
    \begin{tabular}{c | l}
        $R(s)$ & Segnale di riferimento                    \\ \hline
        $E(s)$ & Funzione di errore                        \\ \hline
        $K(s)$ & Funzione di trasferimento del controllore \\ \hline
        $U(s)$ & Ingresso al sistema                       \\ \hline
        $G(s)$ & Funzione di trasferimento del sistema     \\ \hline
        $D(s)$ & Disturbo in uscita                        \\ \hline
        $N(s)$ & Rumore nel sistema di feedback            \\ \hline
        $Y(s)$ & Uscita del sistema
    \end{tabular}
    \caption{Elementi del sistema}
    \label{tab:elementi_sistema}
\end{table}

Si definisce inoltre la funzione di trasferimento ad anello aperto, assumendo che non ci sia il sistema di retroazione e che non ci sia alcun disturbo in uscita (dunque $E(s) = R(s)$) la seguente:
$$
    F(s) = \frac{Y(s)}{R(s)} = K(s)\cdot G(s)
$$

Trascurando ancora il disturbo in uscita ed il rumore nella catena di retroazione si definisce la funzione di \textit{funzione di sensitività complementare $T(s)$} il rapporto tra l'uscita del sistema e il riferimento, includendo però la catena di retroazione:
$$
    T(s) = \frac{Y(s)}{R(s)} = \frac{G(s)\cdot K(s)}{1 + G(s)\cdot K(s)} =\frac{F(s)}{1+F(s)}
$$

Al fine di studiare le caratteristiche di un sistema controllato bisogna prima accertarsi che questo sia stabile a ciclo chiuso, ciò può essere verificato tracciando il diagramma di Nyquist della funzione di trasferimento del sistema $G(s)$, tale diagramma mostra la funzione di trasferimento sul piano complesso di Gauss.

%Questo forse è l'ultimo esercizio con i parametri in transitorio
\newpage
Si mostra di seguito un esempio di rappresentazione sui diagrammi di Bode e Nyquist di una funzione di trasferimento:
\begin{equation}
    F(s) = \frac{5}{s(s+1)}
    \label{eq:trasferimento_es_1}
\end{equation}

\begin{figure}[h]
    \centering
    \begin{tikzpicture}[]
        \begin{scope}[xscale=7/2,yscale=3/80]
            \tikzset{
                semilog lines/.style={black},
            }
            \UnitedB
            \semilog{-1}{1}{-40}{40}
            \BodeGraph[asymp lines,samples=400]
            {-1:1}{\IntAmp{5}+\POAmpAsymp{1}{1}
            }
            \BodeGraph[samples=400]{-1:1}{\IntAmp{5}+\POAmp{1}{1}}
        \end{scope}
        \begin{scope}[xscale=7/2,yscale=3/90]
            \tikzset{
                semilog lines/.style={black},
            }
            \UniteDegre
            \OrdBode{15}
            \semilog{-1}{1}{-180}{-90}
            \BodeGraph[asymp lines,samples=400]
            {-1:1}{\IntArg{5}+\POArgAsymp{1}{1}
            }
            \BodeGraph[samples=400]{-1:1}{\IntArg{5}+\POArg{1}{1}}
        \end{scope}
    \end{tikzpicture}
    \caption{Diagrammi di Bode della funzione di trasferimento \ref{eq:trasferimento_es_1}}
    \label{fig.amplitude_origin_pole}
\end{figure}

A partire dai diagrammi di Bode è possibile dunque riportare la funzione sul diagramma di Nyquist, la curva deve poi essere parametrizzata in funzione di $\omega$, come si può notare nel grafico sottostante sono stati riportati due punti specifici $\omega = 1.5,\ \omega = 2$.
\begin{figure}[h]
    \centering
    \begin{tikzpicture}[
            scale=2
        ]
        \begin{scope}
            \tikzset{
                Nyquist grid/.style={black},
                Nyquist label axis/.style={very thick,blue}
            }

            \NyquistGraph[smooth,samples=200]{0.1:2}
            {\IntAmp{5}+\POAmpAsymp{1}{1}}
            {\IntArg{5}+\POArg{1}{1}}

            \NyquistPoint*[]{1.5/left,2/left}
            {\IntAmp{5}+\POAmpAsymp{1}{1}}
            {\IntArg{5}+\POArg{1}{1}}

            \def\valgridNx{0.2}
            \def\valgridNy{0.2}
            \NyquistGrid

        \end{scope}

    \end{tikzpicture}
    \caption{Diagramma di Nyquist della funzione \ref{eq:trasferimento_es_1}}
\end{figure}

Si richiede in questo esercizio di verificare che la funzione rispetti i requisiti del sistema al transitorio, in particolare si vuole studiare la risposta
al gradino del sistema a ciclo chiuso, per fare ciò si approssima il sistema ad
uno più semplice, si vedrà come determinare se del primo o del secondo ordine.

Per prima cosa si determina la pulsazione critica del sisema, ovvero quella
pulsazione per la quale il modulo della funzione $F(s)$ sia pari ad 1, ovvero pari a 0 nel diagramma di Bode ($\log(1)=0$), osservando il diagramma si vede che
$\omega_c \simeq \SI{2}{\radian/\second}$, è quindi ora necessario determinare il margine di
fase per discernere se il sistema a ciclo chiuso approssimato
è del primo o del secondo ordine, in questo caso il margine di fase è circa
$35^\circ$, dunque la funzione di trasferimento $T_a(s)$ approssimata sarà
del secondo ordine con uno smorzamento di circa $\zeta = 0.35$

\begin{figure}[h]
    \centering
    \begin{tikzpicture}[]
        \begin{scope}[xscale=7/2,yscale=3/80]
            \tikzset{
                semilog lines/.style={black},
            }
            \UnitedB
            \semilog{-1}{1}{-40}{40}
            \BodeGraph[asymp lines,samples=400]
            {-1:1}{\SOAmpAsymp{1}{0.35}{2}}
            \BodeGraph[samples=400]{-1:1}{\SOAmp{1}{0.35}{2}}
            \BodeGraph[color=red]{-1:1}{20*log10(sqrt(25*t**2+(5-5t**2)**2)/((7*t-2*t**3)**2 + (t**4-8*t**2+6)**2))}
        \end{scope}
    \end{tikzpicture}
\end{figure}
