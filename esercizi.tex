\documentclass[10pt,oneside]{article}

\usepackage[a4paper, total={7in, 10in}]{geometry}
\usepackage{ucs}
\usepackage[utf8]{inputenc}
\usepackage{amsmath}
\usepackage{amsfonts}
\usepackage{amssymb}
\usepackage{caption}
\usepackage[italian]{babel}
\usepackage{fontenc}
\usepackage{graphicx}
\usepackage{float}

\usepackage[]{hyperref}
\usepackage{pgfplots} % Plot dei grafici
\usepackage{bodegraph} %pacchetto per diagrammi di Bode
\tikzset{gnuplot def/.append style={prefix={PDF/PDF/}},prefix={PDF/}}

\pgfplotsset{compat=1.18}
% Riduzione tempi di compilazione esternalizzando le figure in pdf separati
\usepgfplotslibrary{external}
%\usetikzlibrary{pgfplots.external}
\tikzexternalize[prefix=PDF/, optimize command away=\includepdf]
%\tikzset[gnuplot def/.append style={prefix={}}]
\usetikzlibrary{shapes,arrows}

\usetikzlibrary{patterns} % libreria per disegnare i pattern

\usepackage{siunitx}

\author{Daniele Olivieri}
\date{18/11/22}

\begin{document}

\tikzstyle{block} = [draw, rectangle,
minimum height=3em, minimum width=6em]
\tikzstyle{sum} = [draw, circle, node distance=1cm]
\tikzstyle{input} = [coordinate]
\tikzstyle{output} = [coordinate]
\tikzstyle{pinstyle} = [pin edge={to-,thin,black}]





\section{Specifiche a regime}

\begin{figure}[h]
    \begin{tikzpicture}[auto, node distance=2cm,>=latex']
        % We start by placing the blocks
        \node [input, name=input] at (0,0) {};
        \node [sum] (sum) at (1,0) {};
        \node [block] (controller) at (3.5,0) {$K(s)$};
        \node [block] (system) at (6.5,0) {$G(s)$};
        % We draw an edge between the controller and system block to
        % calculate the coordinate u. We need it to place the measurement block.
        \draw [->] (controller) -- node[name=u] {$U(s)$} (system);
        \node [sum, ] (disturbance) at (8.5,0) {};
        %\node [input, above of=disturbance] (g_d) {$G_d(s)$};
        \node [input] (dist_input) at (8.5,1) {$D(s)$};
        \node [output] (out2) at (9,0) {};
        \node [sum, ] (noise) at (9,-1.5) {};
        \node [output, right of=disturbance] (output) {};
        \node [input] (noise_in) at (10,-1.5) {};
        %\node [block, below of=u] (measurements) {$GAAA(s)$};


        % Once the nodes are placed, connecting them is easy.
        \draw [->] (input) -- node {$R(s)$} (sum);
        \draw (input) -- node[pos=0.9,yshift=-13] {$+$}(sum);
        \draw [->] (sum) -- node {$E(s)$} (controller);
        \draw [->] (system) -- node[pos=0.9,yshift=-13] {$+$} (disturbance);
        \draw [->] (disturbance) -- node [name=y,xshift=14] {$Y(s)$}(output);

        %\draw [->] (g_d) -- node {$Y_d$} (disturbance);
        \draw [draw, ->] (dist_input) node[xshift=-13,yshift=-5]{$D(s)$} -- node[pos=0.8,xshift=-13] {$+$} (disturbance);
        %\draw [->] (y) |- (measurements);
        \draw [->] (out2) -- node[pos=0.9,xshift=-13] {$+$} node{}(noise);
        %\draw [->] (noise_in) -- node{$N(s)$}(noise);
        \draw [->] (noise) -| node[pos=0.95,xshift=13] {$-$}  node [near end] {} (sum);
        \draw [->] (noise_in) -- node[pos=1] {$+$}  node [label={$N(s)$}] {} (noise);
    \end{tikzpicture}
    \caption[]{Sistema generico}
    \label{fig:sistema_generico}
\end{figure}

In figura \ref{fig:sistema_generico} è possibile visualizzare lo schema elementare di un sistema generico con controllore in retroazione, si indicano in tabella \ref{tab:elementi_sistema} i riferimenti ai vari componenti del sistema, tutte le funzioni si assumono presentate nel dominio di Laplace.
\begin{table}[h]\centering
    \begin{tabular}{c | l}
        $R(s)$ & Segnale di riferimento                    \\ \hline
        $E(s)$ & Funzione di errore                        \\ \hline
        $K(s)$ & Funzione di trasferimento del controllore \\ \hline
        $U(s)$ & Ingresso al sistema                       \\ \hline
        $G(s)$ & Funzione di trasferimento del sistema     \\ \hline
        $D(s)$ & Disturbo in uscita                        \\ \hline
        $N(s)$ & Rumore nel sistema di feedback            \\ \hline
        $Y(s)$ & Uscita del sistema
    \end{tabular}
    \caption{Elementi del sistema}
    \label{tab:elementi_sistema}
\end{table}

Si definisce inoltre la funzione di trasferimento ad anello aperto, assumendo che non ci sia il sistema di retroazione e che non ci sia alcun disturbo in uscita (dunque $E(s) = R(s)$) la seguente:
$$
    F(s) = \frac{Y(s)}{R(s)} = K(s)\cdot G(s)
$$

Trascurando ancora il disturbo in uscita ed il rumore nella catena di retroazione si definisce la funzione di \textit{funzione di sensitività complementare $T(s)$} il rapporto tra l'uscita del sistema e il riferimento, includendo però la catena di retroazione:
$$
    T(s) = \frac{Y(s)}{R(s)} = \frac{G(s)\cdot K(s)}{1 + G(s)\cdot K(s)} =\frac{F(s)}{1+F(s)}
$$


\section{Progettazione di un controllore}

Sia dato un sistema come quello in figura \ref{fig:sistema_generico}, la
cui funzione di trasferimento è:
$$
    G(s) = \frac{30}{(s+10)(s+3)} = \frac{1}{\left(\frac{s}{10}+1\right)\left(\frac{s}{3}+1\right)}
$$
si vuole
progettare un controllore $K(s)$ in modo tale che il sistema a ciclo chiuso
rispetti le seguenti specifiche:
\begin{itemize}
    \item $e(t)|_{\infty} = 0 \ |\ r(t) = 1(t) $ errore a regime nullo con ingresso a gradino $\rightarrow K_1(s) = \frac{1}{s}$
    \item Attenuazione $ATT \geq 95\%\ \omega \in [100\ +\infty[\ \rightarrow |F(j\omega_d)| \geq \SI{-26}{\deci\bel}$
    \item Sovraelongazione $s\leq 20\%$
    \item Tempo di assestamento $t_{s,1\%} \leq 0.5 s$
\end{itemize}

Il primo procedimento è quello di tradurre le specifiche nei diagrammi di Bode,
innanzitutto per annullare l'errore con ingresso a graino è necessario avere
un polo nell'origine, dopo essersi assicurati che questo non sia già presente nel sistema è possibile aggiungere il primo elemento del controllore parziale $K'(s) = \frac{1}{s}$.

Garantire un'attenuazione del 95\% significa avere un sistema che abbia modulo pari a $1-0.95$ con ingresso unitario, superata la pulsazione critica si assume che la funzione $F(s)$ e la funzione $T(s)$ abbiano lo stesso andamento, dunque
è sufficiente inserire nel piano di Bode dei moduli un'area di non attraversammento della $F(s)$ per pulsazioni maggiori di \SI{100}{\radian/\second} e moduli maggiori di $20\log_{10}(1-0.95) = \SI{-26}{\deci{\bel}}$.

La sovraelongazione $s$ ci dà informazioni riguardo la tipologia di funzione approssimante se del primo o secondo ordine e il margine di fase da rispettare,
dalla teoria si sa che dato un valore massimo di sovraelongazione (ad esempio 20\%)
$$
    s = e^{\frac{-\pi \zeta}{\sqrt{1-\zeta^2}}} \leq 20\% \Rightarrow \zeta \geq \sqrt{\frac{\ln s^2}{\pi^2 + \ln s^2}}
$$
in questo caso $\zeta \geq 0.456$, con buona approssimazione si stima il margine
di fase desiderato pari a $100/\zeta = 45.6^\circ$, di conseguenza essendo tale margine minore di \SI{60}{\degree} si approssima il sistema a ciclo chiuso ad uno del secondo ordine del tipo
$$
    T_a(s) =  \frac{1}{1+\frac{2\zeta}{\omega_n}s + \frac{s^2}{\omega_n^2}}
$$
la curva inviluppo della risposta ha un andamento del tipo
$$
    e^{-\zeta \omega_n t_s}
$$
con $\omega_n$ la pulsazione naturale e $t_s$ il tempo di assestamento, se si considera ad esempio un assestamento all'1\% in un tempo di \SI{0.5}{\second} come in questo caso si ricava la pulsazione naturale
$$
    e^{-\zeta \omega_n t_s} \leq 1\% \Rightarrow -\zeta\omega_n t_s \leq \ln{1\%} = -4.6 \Rightarrow \omega_n \geq \frac{4.6}{\zeta t_s} \geq \frac{4.6}{0.456\cdot 0.5} \geq \SI{20}{\radian/\second}
$$

Ricavate tutte le specifiche è possibile iniziare la progettazione del controllore.





\end{document}
